\documentclass[9pt]{beamer}

%~~~~~~~~~~~~~~~~~~~~~~~~~~~~~~~~~~~~~~~~~~~~~~~~~~~~~~~~~~~~~~~~~~~~~~~~~~~~~~
% Use roboto Font (recommended)
%\usepackage[sfdefault]{roboto}
\usepackage[math]{iwona}
\usepackage[utf8]{inputenc}
\usepackage[T1]{fontenc}
%~~~~~~~~~~~~~~~~~~~~~~~~~~~~~~~~~~~~~~~~~~~~~~~~~~~~~~~~~~~~~~~~~~~~~~~~~~~~~~

%~~~~~~~~~~~~~~~~~~~~~~~~~~~~~~~~~~~~~~~~~~~~~~~~~~~~~~~~~~~~~~~~~~~~~~~~~~~~~~
% Define where theme files are located. ('/styles')
\usepackage{styles/fluxmacros}
\usefolder{styles}
% Use Flux theme v0.1 beta
% Available style: asphalt, blue, red, green, gray 
\usetheme[style=asphalt]{flux}
%~~~~~~~~~~~~~~~~~~~~~~~~~~~~~~~~~~~~~~~~~~~~~~~~~~~~~~~~~~~~~~~~~~~~~~~~~~~~~~

%~~~~~~~~~~~~~~~~~~~~~~~~~~~~~~~~~~~~~~~~~~~~~~~~~~~~~~~~~~~~~~~~~~~~~~~~~~~~~~
% Extra packages for the demo:
\usepackage{booktabs}
\usepackage{colortbl}
\usepackage{ragged2e}
\usepackage{schemabloc}
%~~~~~~~~~~~~~~~~~~~~~~~~~~~~~~~~~~~~~~~~~~~~~~~~~~~~~~~~~~~~~~~~~~~~~~~~~~~~~~

%~~~~~~~~~~~~~~~~~~~~~~~~~~~~~~~~~~~~~~~~~~~~~~~~~~~~~~~~~~~~~~~~~~~~~~~~~~~~~~
% Paquetes adicionales
\usepackage[spanish,mexico]{babel}
\usepackage{multirow}
\usepackage{caption}
\usepackage{xltabular}
\usepackage{subcaption}
\usepackage[flushleft]{threeparttable}
\usepackage{listings}
\usepackage{xcolor}

\definecolor{codegreen}{rgb}{0,0.6,0}
\definecolor{codegray}{rgb}{0.5,0.5,0.5}
\definecolor{codepurple}{rgb}{0.58,0,0.82}
\definecolor{backcolour}{rgb}{0.95,0.95,0.92}

\lstdefinestyle{mystyle}{
    backgroundcolor=\color{backcolour},   
    commentstyle=\color{codegreen},
    keywordstyle=\color{magenta},
    numberstyle=\tiny\color{codegray},
    stringstyle=\color{codepurple},
    basicstyle=\ttfamily\footnotesize,
    breakatwhitespace=false,         
    breaklines=true,                 
    captionpos=b,                    
    keepspaces=true,                 
    numbers=left,                    
    numbersep=5pt,                  
    showspaces=false,                
    showstringspaces=false,
    showtabs=false,                  
    tabsize=2,
    xleftmargin=.2in,
    xrightmargin=.2in
}
\lstset{style=mystyle}
%~~~~~~~~~~~~~~~~~~~~~~~~~~~~~~~~~~~~~~~~~~~~~~~~~~~~~~~~~~~~~~~~~~~~~~~~~~~~~~

%~~~~~~~~~~~~~~~~~~~~~~~~~~~~~~~~~~~~~~~~~~~~~~~~~~~~~~~~~~~~~~~~~~~~~~~~~~~~~~
% P O R T A D A
% Título de la tesis
\title{Título de la tesis}

% Descomenta lo que vas a presentar
\subtitle{Defensa de Tesis}
%\subtitle{Avance de Tesis}
%\subtitle{Anteproyecto de Tesis}

% Descomenta el programa al que perteneces
%\institute{Maestría en Ingeniería para la Innovación Tecnológica}
\institute{Doctorado en Ingeniería para la Innovación Tecnológica}
%\institute{Ingeniería en}

% Pon tu nombre completo sin grado (Ing, Lic, etc), a los miembros del comité pon el nombre completo junto con su grado (Dr., M.C., etc) y con paréntesis indica su rol en el comité, ya sea (Director) o (Co-Director), los demás miembros quedan en blanco, pueden ser un director y hasta dos co-directores
\author{Nombre completo del estudiante \and 
    Comité de Tesis \and
    Dr. Nombre del director de tesis (Director)\and
    Dr. Nombre del co-director de tesis (Co-Director)\and
    Dr. Nombre del miembro del comité \and
    Dr. Nombre del miembro del comité \and
    Dr. Nombre del miembro del comité}
\date{\today}
%~~~~~~~~~~~~~~~~~~~~~~~~~~~~~~~~~~~~~~~~~~~~~~~~~~~~~~~~~~~~~~~~~~~~~~~~~~~~~~

\begin{document}
% Generate title page
\justifying
\titlepage
\begin{frame}[allowframebreaks]{Contenido}
 \tableofcontents
\end{frame}

\section{Estructura de la presentación}
\begin{frame}[allowframebreaks,fragile]{Estructura de la presentación}
La estructura de una presentación del anteproyecto, avance o defensa de tesis debe ser clara, concisa y organizada para cubrir los puntos clave de la investigación de manera efectiva. Esta debe de incluir las siguientes secciones:

\begin{enumerate}
    \item Título e Introducción:
    \begin{enumerate}
        \item Título de la Tesis y datos del autor (nombre, programa de estudios, nombre del director de tesis, fecha).
        \item Objetivo de la Presentación: Explicar brevemente el propósito de la defensa.
        \item Contexto o Introducción General: Mencionar el problema de investigación, su relevancia y cómo se sitúa en el campo de estudio.
    \end{enumerate}
    \item Planteamiento del Problema:
    \begin{enumerate}
        \item Describir claramente el problema que aborda la tesis.
        \item Explicar por qué es importante resolver o estudiar este problema.
        \item Presentar la pregunta de investigación principal.
    \end{enumerate}
    \item Objetivos de la Investigación:
    \begin{enumerate}
        \item Objetivo General: Resumir el propósito principal de la tesis.
        \item Objetivos Específicos: Enumerar los pasos específicos para lograr el objetivo general.
    \end{enumerate}
    \item Marco Teórico:
    \begin{enumerate}
        \item Breve mención de las teorías, conceptos y estudios previos que sustentan la investigación.
        \item Destacar las bases científicas o teóricas más relevantes para el problema estudiado.
    \end{enumerate}
    \item Metodología:
    \begin{enumerate}
        \item Explicar el enfoque de investigación (cuantitativo, cualitativo, mixto).
        \item Describir los métodos y técnicas utilizados para la recolección y análisis de datos.
        \item Explicar el diseño de la investigación y los instrumentos empleados.
    \end{enumerate}
    \item Resultados:
    \begin{enumerate}
        \item Presentar los hallazgos principales de la investigación, utilizando gráficos, tablas o diagramas para facilitar la comprensión.
        \item Explicar los datos relevantes, resaltando cómo responden a la pregunta de investigación.
    \end{enumerate}
    \item Análisis y Discusión:
    \begin{enumerate}
        \item Interpretar los resultados obtenidos y relacionarlos con el marco teórico.
        \item Comparar los hallazgos con estudios previos y discutir su implicación.
        \item Identificar cualquier limitación en los resultados y sugerir áreas para investigaciones futuras.
    \end{enumerate}
    \item Cronograma de la Investigación:
    \begin{enumerate}
        \item Descripción del Cronograma: Presentar una línea de tiempo o diagrama de Gantt que muestre las fases de la investigación (búsqueda bibliográfica, diseño metodológico, recolección de datos, análisis, redacción).
        \item Explicar brevemente el proceso y cómo se distribuyó el tiempo para cada actividad, demostrando una organización eficiente.
    \end{enumerate}
    \item Conclusiones:
    \begin{enumerate}
        \item Resumir las conclusiones más importantes de la investigación en relación con los objetivos y la pregunta de investigación.
        \item Mencionar la contribución al campo de estudio o posibles aplicaciones prácticas.
    \end{enumerate}
    \item Trabajo futuro (Opcional):
    \begin{enumerate}
        \item Sugerir posibles líneas de investigación futura o aplicaciones de los resultados.
    \end{enumerate}
    \item Referencias:
    \begin{enumerate}
        \item Incluir una Diapositiva de Referencias: Mostrar una lista de las fuentes más importantes que sustentaron el marco teórico y la metodología.
        \item Asegurarse de utilizar el formato de citación APA.
    \end{enumerate}
    \item Agradecimientos (Opcional):
    \begin{enumerate}
        \item Agradecer a aquellas personas o instituciones que contribuyeron significativamente a la realización de la tesis.
    \end{enumerate}
    \item Preguntas del Comité de Tesis:
    \begin{enumerate}
        \item Preparación para las Preguntas: Al terminar la presentación el comité realizará preguntas para la evaluación del trabajo realizado.
    \end{enumerate}
\end{enumerate}
\pagebreak
Consejos:

\begin{enumerate}
    \item [] Tiempo: La presentación debe ser concisa de 20 a 30 minutos.
    \item [] Diapositivas Visuales: Usar gráficos, tablas y diagramas que ayuden a resumir y clarificar la información.
    \item [] Ensayar: Practicar la presentación para mejorar el dominio del tema y la claridad de la exposición.
\end{enumerate}
\end{frame}
\section{Planteamiento del Problema}
\begin{frame}[allowframebreaks,fragile]{Planteamiento del Problema}
El planteamiento del problema es una sección que describe de manera clara y detallada el problema específico que se va a investigar. Este es el punto de partida de la investigación y define el enfoque que guiará todos los esfuerzos. En esta sección, el investigador justifica la relevancia del problema, establece el contexto en el que se encuentra y explica por qué es necesario abordarlo, especialmente en el ámbito de la ingeniería.

\begin{enumerate}
    \item Contextualización del Problema:
    Comienza proporcionando un contexto amplio sobre el área de estudio y el entorno en el que se sitúa el problema.

    Incluye una descripción general de los avances o carencias en la tecnología o el campo de estudio, mencionando posibles necesidades industriales, problemas prácticos, desafíos económicos, o metas científicas que aún no han sido alcanzadas.

    \item Descripción Específica del Problema:

    Define de forma precisa el problema central de la investigación. Esto puede incluir aspectos técnicos o científicos, limitaciones de los métodos o tecnologías actuales, o una necesidad específica que aún no ha sido satisfecha.

    El problema debe ser claramente delimitado, evitando ambigüedades y enfocándose en el aspecto particular que la tesis busca resolver o mejorar.

    \item Justificación de la Relevancia del Problema:

    Explica por qué el problema es relevante desde el punto de vista científico, técnico o social, y cómo su resolución contribuirá al campo de estudio o tendrá aplicaciones prácticas.

    Resalta el impacto potencial que tendría abordar este problema, como mejorar la eficiencia, precisión, seguridad, costos o innovación en una tecnología o proceso de ingeniería.

    \item Consecuencias de No Abordar el Problema:

    Expón los efectos negativos que implicaría no resolver el problema identificado, ya sea en términos de pérdidas económicas, limitaciones tecnológicas, riesgo de seguridad, o falta de competitividad en el sector.

    Esta parte ayuda a fortalecer la necesidad de la investigación y a subrayar la importancia de encontrar una solución.

    \item Declaración del Problema:

    Redacta la formulación del problema de manera clara y concisa. Puede ser en forma de pregunta de investigación o de un enunciado que exprese la necesidad de desarrollo o de solución.

    La declaración debe reflejar el núcleo de la investigación y guiar el resto de los capítulos de la tesis.

    \item Objetivos del Planteamiento del Problema (Opcional):

    Finaliza el planteamiento con una breve introducción a los objetivos generales de la investigación, que posteriormente se desarrollarán en la sección de objetivos. Esta conexión prepara el terreno para mostrar cómo se abordará el problema planteado.
\end{enumerate}

Consideraciones:

Un planteamiento de problema efectivo en una tesis de posgrado en ingeniería debe ser claro y directo, evitando descripciones excesivas pero brindando el suficiente detalle para justificar la investigación.

Es recomendable usar términos técnicos y lenguaje formal, adecuado al nivel académico y a la especialización de los lectores.


El planteamiento del problema debe dejar claro el desafío técnico o científico y su relevancia, estableciendo así la base para el desarrollo de los objetivos y la metodología de la investigación.
\end{frame}
\section{Objetivos}
\begin{frame}[allowframebreaks,fragile]{Objetivos}
Los objetivos se definen para guiar y delimitar el alcance del proyecto, estableciendo metas claras que orienten el desarrollo de la investigación. 

Los objetivos dividen el propósito de la investigación en tareas alcanzables, y suelen dividirse en un objetivo general y varios objetivos específicos.

Pasos para  efinir los Objetivos en una Tesis de Posgrado en Ingeniería:

\begin{enumerate}
    \item Definir el Objetivo General:

    El objetivo general responde a la pregunta: ¿Qué pretende lograr esta investigación de manera amplia?

    Describe el propósito principal del proyecto, planteado de manera precisa y concisa.

    Ejemplos de verbos comunes para el objetivo general: ``desarrollar'', ``diseñar'', ``evaluar'', ``proponer'', ``optimizar'', ``implementar''.

    En ingeniería, el objetivo general suele estar orientado a la creación, mejora, o evaluación de una tecnología, método o proceso.

    \item Establecer Objetivos Específicos:

    Los objetivos específicos desglosan el objetivo general en tareas concretas y pasos secuenciales, enfocándose en aspectos específicos que deben lograrse para cumplir el objetivo general.

    Cada objetivo específico debe ser claro, alcanzable y medible. En una tesis de ingeniería, estos pueden incluir la investigación teórica, el diseño de prototipos, la implementación de pruebas, y la evaluación de resultados.

    Usualmente se utilizan verbos como ``analizar'', ``identificar'', ``probar'', ``validar'', ``simular'', ``medir'', que reflejan las etapas del desarrollo experimental y de validación.

    \item Enfocar los Objetivos en resultados medibles:

    Dado que una tesis de ingeniería suele involucrar mediciones, pruebas y análisis de resultados, los objetivos deben estar orientados hacia metas cuantificables o medibles.

    Esto permite verificar el cumplimiento de cada objetivo específico al concluir la investigación, con resultados concretos.

    \item Conectar los Objetivos con el Planteamiento del Problema:

    Los objetivos deben resolver o avanzar en la solución del problema planteado, demostrando cómo la investigación aborda necesidades concretas o limitaciones del conocimiento o la tecnología actuales.

    Para cada objetivo específico, es recomendable verificar que esté alineado con una necesidad o aspecto mencionado en el planteamiento del problema.
\end{enumerate}

Consejos adicionales para definir Objetivos en una Tesis de Ingeniería:

\begin{enumerate}
    \item Asegúrate de que los objetivos sean realistas y alcanzables dentro del marco de la tesis.
    \item Escribe cada objetivo de manera concisa y específica, evitando la ambigüedad.
    \item Revisa que los objetivos cubran todas las fases de la investigación, desde la teoría y el diseño hasta la validación y análisis de resultados.
    \item Al estructurar los objetivos de esta forma, se logra que la investigación sea metódica y organizada, avanzando de manera lógica desde los conceptos iniciales hasta la implementación y validación del trabajo.
\end{enumerate}    
\end{frame}
\section{Marco Teórico}
\begin{frame}[allowframebreaks,fragile]{Marco Teórico}
En el caso de una tesis de posgrado en Ingeniería, el marco teórico toma un enfoque aún más profundo y especializado para demostrar un conocimiento avanzado del tema. Esto implica una revisión exhaustiva de los fundamentos científicos, las tecnologías específicas y las investigaciones más recientes y relevantes en el área.

\begin{enumerate}
    \item Profundización en Teorías y Modelos Avanzados:

    Se requieren teorías y modelos más complejos que permitan abordar problemas específicos de ingeniería. Esto puede incluir modelos matemáticos avanzados, simulaciones, o algoritmos detallados.

   \item Estado del Arte y Últimas Tendencias:

    El marco teórico en una tesis de posgrado incluye una revisión del estado del arte en el tema, analizando las soluciones tecnológicas más avanzadas y recientes.

    Identificar y discutir investigaciones de vanguardia permite definir claramente las limitaciones actuales y el aporte innovador de la tesis.

    \item Fundamentación Tecnológica y Algorítmica:

    En una investigación de posgrado, es esencial desarrollar una descripción detallada de las tecnologías y algoritmos empleados en el proyecto.

    \item Revisión Crítica y Justificación del Enfoque:

    A diferencia de una tesis de grado, el nivel de posgrado exige una revisión crítica de las teorías y técnicas seleccionadas, justificando por qué se eligieron ciertos métodos y no otros.

    \item Modelado Matemático y Simulación:

    Es común incluir modelado matemático para representar el sistema o fenómeno estudiado, así como simulaciones que permitan validar teorías antes de implementarlas.

    \item Base para la Metodología:

    El marco teórico en un posgrado no solo da contexto, sino que establece la base para la metodología. Los conceptos y teorías presentados deben conectarse claramente con los métodos y técnicas que se utilizarán en la investigación.
\end{enumerate}
\end{frame}
\section{Metodología}
\begin{frame}[allowframebreaks,fragile]{Metodología}
La metodología en una tesis de posgrado en ingeniería describe el enfoque, los métodos y las técnicas específicas que se utilizarán para alcanzar los objetivos de la investigación y resolver el problema planteado. La metodología es esencial para dar claridad al proceso de investigación, detallando cómo se llevarán a cabo cada etapa del estudio, las herramientas y recursos necesarios, así como los criterios de validación y análisis.

Estructura de la Metodología:

\begin{enumerate}
    \item Enfoque de la investigación:

    Define el tipo de enfoque (experimental, cuantitativo, cualitativo, mixto) que mejor responde a la naturaleza del problema.

    En ingeniería, el enfoque suele ser experimental y cuantitativo, ya que busca medir y analizar datos concretos para evaluar el desempeño de soluciones o tecnologías.

    \item Diseño experimental o estructura de trabajo:

    Describe el diseño experimental o la estructura de trabajo en términos de fases o etapas.

    Detalla cómo se desarrollará la investigación de manera organizada, desde la revisión inicial hasta la evaluación final.

    Estas fases pueden incluir análisis teórico, simulaciones, desarrollo de prototipos, pruebas experimentales, optimización, etc.

    \item Herramientas, equipos y recursos:

    Especifica el hardware, software, equipos de laboratorio, y recursos materiales necesarios para la investigación.

    Por ejemplo, en ingeniería, esto podría incluir dispositivos de medición, sensores, microcontroladores, software de simulación, etc.

    También se debe mencionar cualquier recurso humano adicional (asistentes de investigación, expertos, etc.) que apoyará el desarrollo.

    \item Procedimientos técnicos y pasos de desarrollo:

    Desglosa las etapas específicas y procedimientos técnicos, como la construcción de prototipos, configuración de equipos, recolección de datos y análisis estadístico.

    Cada paso debe estar bien explicado para que sea reproducible, especialmente en los aspectos de configuración de parámetros, ejecución de pruebas y toma de medidas.

    \item Simulaciones y modelado (si aplica):

    Si la investigación requiere simulaciones o modelado matemático, explica el software y las técnicas específicas utilizadas, como el diseño asistido por computadora (CAD), análisis de elementos finitos (FEA), Matlab, Python, etc.

    Describe el propósito de estas simulaciones y cómo se alinean con los objetivos específicos.

    \item Pruebas experimentales y protocolos de validación:

    Define los experimentos, las pruebas de campo o de laboratorio, y el método de recolección de datos.

    Especifica los criterios de validación que utilizarás para evaluar los resultados, que pueden incluir pruebas de precisión, eficiencia, seguridad, fiabilidad, etc.

    \item Análisis de datos y métodos de evaluación:

    Describe los métodos para procesar y analizar los datos obtenidos, como técnicas estadísticas, análisis de error, o métodos de interpretación gráfica.

    En una investigación de ingeniería, los datos suelen evaluarse en función de criterios de rendimiento, eficiencia o precisión, por lo que los métodos de evaluación deben ajustarse a estos parámetros.

    \item Limitaciones y alcance de la Metodología:

    Describe cualquier limitación del enfoque o los métodos empleados, explicando las posibles restricciones de la investigación y el alcance que se espera lograr.

    Esta parte es importante para comprender hasta dónde llegan los resultados y en qué contextos podrían aplicarse.
\end{enumerate}

Consejos Adicionales para Definir la Metodología en una Tesis de Ingeniería:

\begin{enumerate}
    \item Asegúrate de que cada método o técnica esté alineado con un objetivo específico, mostrando cómo contribuye a resolver el problema planteado.
    \item Describe con precisión los equipos, software y protocolos de seguridad, ya que estos detalles son cruciales en trabajos de ingeniería.
    \item Sé claro y detallado en cada paso del procedimiento para que otros investigadores puedan replicar tu metodología.
\end{enumerate}

La metodología debe demostrar un proceso sólido, estructurado y científicamente válido, que no solo permita alcanzar los objetivos de la investigación, sino que también respalde la credibilidad y aplicabilidad de los resultados.
\end{frame}
\section{Resultados}
\begin{frame}[allowframebreaks,fragile]{Resultados}
La sección de resultados es fundamental, ya que presenta los hallazgos obtenidos de las pruebas, experimentos o simulaciones que se realizaron siguiendo la metodología establecida. Esta sección debe describir y analizar los datos de manera clara y objetiva, destacando cómo estos resultados cumplen (o no) con los objetivos planteados y cómo responden al problema de investigación.

Estructura de la Sección de Resultados:

\begin{enumerate}
    \item Presentación de Datos Obtenidos:

    Organiza los datos recolectados en tablas, gráficos o figuras, que permitan visualizar la información de forma clara y resumida.

    Incluye únicamente datos relevantes y omite información redundante.

    Cada tabla o figura debe tener un título claro y una breve explicación que permita entenderla de manera independiente.

    \item Descripción de los Resultados:

    Describe los resultados observados sin interpretarlos. En esta etapa, el enfoque debe estar en explicar qué muestran los datos de manera objetiva.

    Organiza esta descripción de acuerdo a los objetivos específicos, para que el lector pueda ver claramente cómo se cumplieron cada uno de ellos.

    Puedes organizar esta sección en subapartados para cada experimento, simulación o fase del proyecto, según sea el caso.

    \item Análisis e interpretación:

    Analiza los resultados, interpretando su significado en el contexto del problema de investigación y los objetivos planteados.

    Identifica patrones, tendencias o relaciones relevantes entre las variables. Explica cómo estos hallazgos se relacionan con el desarrollo del sistema o tecnología investigada.

    En ingeniería, esto podría incluir comparaciones con estándares de la industria, con estudios previos o con parámetros de rendimiento esperados.

    \item Comparación con la literatura o estudios previos:

    Compara los resultados obtenidos con estudios similares o literatura previa para ver cómo se posicionan tus hallazgos en el contexto del campo de estudio.

    Esto ayuda a validar tus resultados y demuestra que el trabajo tiene una base científica sólida y contribuye al conocimiento existente en la materia.

    \item Validación y discusión de los resultados:

    Expón la validez de tus hallazgos explicando las pruebas de control, análisis estadístico, o métodos de validación que se aplicaron.

    Discute las posibles causas de cualquier desviación, error o anomalía observada, explicando cómo afectaron los resultados y en qué medida impactan en la conclusión de la investigación.

    En esta parte puedes mencionar si los resultados son reproducibles o si existen limitaciones que afectaron la precisión de los datos.

    \item Presentación de resultados clave para conclusiones:

    Resalta los resultados más importantes que serán la base para las conclusiones finales de la tesis.

    Los resultados clave deben conectarse directamente con el planteamiento del problema y con los objetivos generales y específicos.
\end{enumerate}

Consejos para Redactar la Sección de Resultados en una Tesis de Ingeniería

\begin{enumerate}
    \item Usa un lenguaje preciso y evita interpretaciones subjetivas; presenta los resultados de manera objetiva.
    \item Asegúrate de que cada dato esté respaldado visualmente (tablas o gráficos) y acompaña cada gráfico de una breve descripción y análisis.
    \item Evita sobrecargar la sección con demasiados datos. Resalta lo que es crucial y utiliza anexos para información adicional, si es necesario.
\end{enumerate}

La sección de resultados debe presentar de manera estructurada y visual los logros alcanzados, evidenciando cómo estos contribuyen a la solución del problema de investigación y cumplen con los objetivos planteados.
\end{frame}
\section{Análisis y Discusión}
\begin{frame}[allowframebreaks,fragile]{Análisis y Discusión}
La sección de análisis y discusión es donde se interpretan y contextualizan los resultados, evaluando su relevancia en relación con los objetivos de la investigación y el problema planteado. En esta sección, el investigador conecta los hallazgos con el conocimiento existente, discute su validez y limita su alcance, aportando una visión crítica y detallada de los resultados obtenidos.

Estructura de la Sección de Análisis y Discusión:

\begin{enumerate}
    \item Interpretación de los Resultados:

    Explica el significado de los resultados obtenidos, relacionándolos con cada objetivo específico de la investigación.

    Describe cómo los datos responden al planteamiento del problema y destacan los logros alcanzados o los aspectos en los que los resultados fueron distintos a lo esperado.

    En ingeniería, esto puede incluir una discusión sobre la eficiencia, precisión o desempeño del sistema desarrollado en diferentes condiciones.

    \item Comparación con estudios previos o literatura:

    Compara tus resultados con estudios o investigaciones previas en el campo, evaluando si tus hallazgos son consistentes o presentan diferencias.

    Si tus resultados difieren de lo que otros autores han reportado, justifica estas discrepancias. Esto puede deberse a diferencias en metodologías, condiciones de prueba, o materiales utilizados.

    La comparación ayuda a ubicar tu investigación dentro del contexto más amplio del área de estudio, mostrando su relevancia y originalidad.

    \item Implicaciones prácticas y teóricas:

    Discute las posibles aplicaciones prácticas de los resultados y cómo contribuyen al avance tecnológico o metodológico en el área de ingeniería.

    Reflexiona sobre las implicaciones teóricas de tus hallazgos, aportando a la base de conocimiento existente y sugiriendo cómo podrían influir en futuras investigaciones.

    \item Limitaciones del estudio:

    Expón las limitaciones técnicas o metodológicas que pudieron afectar los resultados, como restricciones en los recursos, errores experimentales, o limitaciones en los equipos.

    Reconocer las limitaciones muestra la transparencia del investigador y ayuda a los lectores a evaluar el alcance de las conclusiones.

    \item Posibles explicaciones para inconsistencias:

    Si algunos resultados no fueron los esperados, sugiere posibles explicaciones, como variaciones en las condiciones de prueba, factores externos, o limitaciones de los métodos.

    Propón recomendaciones para evitar estas inconsistencias en investigaciones futuras.

    \item Propuestas para Trabajos Futuros:

    Con base en los hallazgos y limitaciones identificadas, ofrece recomendaciones o líneas de investigación que podrían expandir o profundizar los resultados obtenidos.

    Sugerir trabajos futuros muestra la proyección de la investigación, señalando áreas que aún pueden ser mejoradas o exploradas.
\end{enumerate}


Consejos para Redactar la Sección de Análisis y Discusión en una Tesis de Ingeniería

\begin{enumerate}
    \item Asegúrate de hacer una interpretación lógica y detallada de los resultados, evitando conclusiones apresuradas.
    \item Evita redundancias con la sección de resultados. En análisis y discusión, céntrate en el significado e implicación de los datos.
    \item Sé claro y crítico. Aborda tanto los éxitos como las limitaciones y áreas de mejora.
    \item Muestra cómo tu investigación contribuye al avance en el campo y por qué es relevante para el desarrollo de nuevas tecnologías, métodos o conocimientos.
\end{enumerate}

La sección de análisis y discusión debe ofrecer una visión profunda y crítica del trabajo, conectando los resultados con el problema inicial, y destacando la relevancia, limitaciones y potenciales futuras aplicaciones de la investigación en ingeniería.
\end{frame}
\section{Cronograma}
\begin{frame}[allowframebreaks,fragile]{Cronograma}
El cronograma es una herramienta de planificación que detalla el tiempo estimado para completar cada una de las actividades de investigación. Este apartado es fundamental para organizar el trabajo y asegurarse de que cada fase del proyecto se lleve a cabo en los tiempos previstos. Un cronograma bien estructurado ayuda tanto a cumplir con los objetivos a tiempo como a visualizar el progreso del proyecto.

Elementos Clave del Cronograma

\begin{enumerate}
    \item Desglose de actividades:

    Enumera todas las actividades y fases principales de la investigación, desde la revisión de la literatura y el desarrollo de la metodología, hasta la experimentación, análisis de datos y redacción del informe final.

    Agrupa actividades relacionadas bajo categorías, como ``Revisión de literatura'', ``Diseño experimental'', ``Implementación'', ``Pruebas y validación'', ``Análisis de resultados'' y ``Redacción y revisión de la tesis''.

    \item Estimación de tiempos:

    Asigna una duración específica para cada actividad, indicando cuánto tiempo se espera que tome cada fase (en días, semanas o meses, según sea conveniente).

    La duración debe ser realista, considerando posibles contratiempos o periodos de revisión.

    \item Orden secuencial y dependencias:

    Presenta las actividades en el orden en el que deben realizarse. Algunas actividades pueden depender de la finalización de otras, por lo que debe haber claridad sobre estas dependencias.

    Por ejemplo, ``Pruebas y validación'' depende de que se haya completado la ``Implementación''.

    \item Formato visual del cronograma:

    La forma más común de presentar un cronograma es a través de un diagrama de Gantt, donde cada actividad se representa como una barra en una línea de tiempo.

    Cada barra muestra el inicio y fin de la actividad, así como su duración, lo que permite visualizar el avance del proyecto en cada etapa.

    \item Hitos o entregables clave:

    Incluye hitos o puntos clave del proyecto, como la finalización de cada fase principal o la entrega de informes parciales, para marcar el progreso y evaluar si el proyecto avanza según lo planificado.

    Estos hitos pueden incluir eventos como ``Finalización de la revisión bibliográfica'', ``Prototipo completado'', ``Pruebas concluidas'', ``Primer borrador de la tesis'', etc.

    \item Flexibilidad y revisión del cronograma:

    El cronograma debe considerar un margen de flexibilidad para abordar posibles retrasos o ajustes necesarios en las fases de la investigación.

    Es útil añadir una ``revisión de cronograma'' periódica para evaluar si los tiempos asignados se están cumpliendo o si es necesario realizar ajustes.
\end{enumerate}

Consejos para Elaborar un Cronograma de Tesis en Ingeniería

\begin{enumerate}
    \item Sé realista: Estima los tiempos basándote en experiencias previas y en los recursos disponibles. Evita subestimar el tiempo que puede llevar una actividad.
    \item Añade un margen de seguridad: Reserva tiempo extra para etapas cruciales, como las pruebas y la redacción, donde suelen presentarse ajustes y correcciones.
    \item Evalúa y ajusta periódicamente: Monitorea el avance y realiza ajustes si se detectan retrasos o cambios en el proyecto.
    \item Usa software de gestión de proyectos: Herramientas como Microsoft Project, Trello o incluso Google Sheets pueden facilitar la creación y seguimiento del cronograma.
\end{enumerate}

El cronograma debe reflejar una planificación estratégica y organizada, demostrando que el proyecto está estructurado de manera eficiente para cumplir con los plazos establecidos.
\end{frame}
\section{Conclusiones}
\begin{frame}[allowframebreaks,fragile]{Conclusiones}

La sección de conclusiones resume los hallazgos más importantes de la investigación, reflexiona sobre su relevancia y explica el cumplimiento de los objetivos planteados. Esta sección es crucial porque cierra la tesis, destacando el valor del trabajo y sus contribuciones al campo de estudio.

En el caso de un avance de tesis, la sección de conclusiones también debe reflejar un análisis provisional de los logros y aprendizajes alcanzados hasta el momento. Aunque la tesis no esté completa, esta sección permite identificar el avance del trabajo y orientar los siguientes pasos de manera clara y organizada.


Estructura de la Sección de Conclusiones:

\begin{enumerate}
    \item Síntesis de los resultados principales:

    Resume los resultados más relevantes, destacando cómo estos abordan los objetivos específicos y el problema de investigación.

    En ingeniería, es común enfatizar los logros técnicos o avances logrados en el desarrollo, validación o aplicación de métodos, prototipos o tecnologías.

    \item Cumplimiento de los Objetivos:

    Explica si se cumplieron los objetivos específicos y el objetivo general de la investigación.

    Aborda cada objetivo brevemente y muestra cómo los resultados logrados los alcanzaron o si existieron limitaciones que afectaron su cumplimiento.

    \item Contribución al Campo de Ingeniería:

    Describe cómo la investigación contribuye al conocimiento o desarrollo en el área específica de ingeniería.

    Esto puede incluir innovaciones en técnicas, mejoras en rendimiento, soluciones a problemas prácticos, o descubrimientos teóricos importantes.

    \item Limitaciones y reflexión crítica:

    Reconoce cualquier limitación que surgió durante la investigación y cómo estas afectaron los resultados.

    Una reflexión crítica muestra la capacidad de autoevaluación del investigador y ofrece un análisis realista sobre los alcances del estudio.

    \item Implicaciones prácticas y aplicaciones futuras:

    Explica cómo podrían aplicarse los resultados en la práctica o cómo benefician al sector industrial, tecnológico, académico, etc.

    Menciona cómo el trabajo realizado puede influir en estudios futuros o en el desarrollo de nuevas tecnologías.

    \item Recomendaciones para Trabajos Futuros:

    Sugerir áreas específicas que aún necesitan investigación o aspectos técnicos que podrían mejorarse.

    Estas recomendaciones muestran la proyección de la investigación hacia el futuro y posibles líneas de continuidad.
\end{enumerate}

Consejos para Redactar la Sección de Conclusiones en una Tesis de Ingeniería

\begin{enumerate}
    \item Sé conciso y directo: Evita repetir toda la información; céntrate en los hallazgos más importantes y en el valor de tu investigación.
    \item Usa un tono reflexivo y objetivo: Muestra el impacto del trabajo, pero evita exagerar los logros.
    \item Evita nuevos datos: Las conclusiones deben basarse en los resultados ya presentados, no incluir datos adicionales.
    \item Enfócate en el futuro: Una buena conclusión cierra el trabajo pero también sugiere cómo otros pueden construir sobre él.
\end{enumerate}

La sección de conclusiones es clave para demostrar cómo el trabajo aborda de manera efectiva el problema planteado y contribuye al avance en el campo de ingeniería.

En caso de un Avance de Tesis, la Sección de Conclusiones se estructura de la siguiente manera: 

\begin{enumerate}
    \item Resumen de Logros Parciales:

    Resume los resultados obtenidos hasta la fecha, resaltando avances clave en el desarrollo del proyecto, como pruebas preliminares, diseño de prototipos, o validación de modelos.

    Esto brinda una visión de cómo el proyecto se está encaminando hacia los objetivos finales.

    \item Cumplimiento de Objetivos Parciales:

    Evalúa qué objetivos específicos han sido cumplidos y qué pasos siguen pendientes.

    Para los objetivos aún no alcanzados, se pueden mencionar las tareas planeadas para lograrlos.
    
    \item Contribuciones temporales al Proyecto:

    Menciona los avances que representan una contribución al campo o a la solución del problema, aunque todavía no sean conclusivos.

    Por ejemplo, podrías incluir descubrimientos preliminares, mejoras iniciales o aportes en metodología que ya estén perfilando el aporte final.

    \item Limitaciones y Retos encontrados:

    Describe cualquier limitación encontrada hasta el momento, señalando si se necesita ajustar el enfoque o la metodología para superarlas en la fase siguiente.

    \item Reflexión sobre el progreso y Plan de Trabajo:

    Reflexiona sobre cómo los avances logrados impactan en el desarrollo global de la tesis y qué ajustes se prevén en la planificación.

    Define los próximos pasos de trabajo, estableciendo las actividades prioritarias y las metas a corto plazo.

    \item Proyecciones y siguientes pasos:

    Describe las actividades específicas que siguen en el proyecto y cómo cada una contribuirá a la conclusión de la tesis.

    Este apartado debe mostrar una hoja de ruta clara que permita visualizar la continuidad del trabajo.
\end{enumerate}

En el contexto de un avance de tesis, las conclusiones ayudan a demostrar el progreso hecho y ofrecen una visión de los próximos pasos necesarios para la culminación del proyecto.
\end{frame}
\section{Referencias}
\begin{frame}[allowframebreaks,fragile]{Referencias}
La sección de referencias es esencial para respaldar el contenido y brindar reconocimiento a las fuentes de información empleadas. Este apartado incluye todas las fuentes citadas a lo largo del trabajo, como artículos científicos, libros, tesis, informes técnicos, normas, sitios web, y cualquier otro material académico o técnico relevante.

Estructura de la Sección de Referencias

\begin{enumerate}
    \item Estilo de Citación:

    Respeta el estilo de citación indicado en los lineamientos del programa. 
    
    Es importante seguir el mismo estilo de forma consistente en toda la presentación.

    \item Tipos de Fuentes Comunes en Ingeniería:
    \begin{itemize}
        \item Artículos científicos: La investigación en ingeniería se apoya en gran medida en artículos revisados por pares. Incluye el nombre de los autores, título del artículo, nombre de la revista, volumen, número, páginas y año de publicación.
        \item Libros: Cuando citas libros de referencia, incluye el autor(es), título, edición (si aplica), editorial, y año de publicación.
        \item Informes técnicos o normativas: Cita reportes técnicos, estándares (como los de IEEE o ISO), o normativas que fundamenten aspectos técnicos o metodológicos.
        \item Patentes: Si citas patentes, incluye el inventor, número de patente, título, y fecha de publicación.
        \item Tesis y documentos académicos: Al citar tesis previas, incluye autor, título, institución, tipo de documento (tesis de maestría o doctoral), y año.
        \item Recursos electrónicos: Para sitios web, manuales en línea, o documentos digitales, incluye URL y la fecha en la que se consultó el recurso.
    \end{itemize}
    \item Uso de Herramientas de Gestión Bibliográfica:

    Herramientas como EndNote, Zotero, Mendeley, o LaTeX pueden facilitar la creación y gestión de las referencias, asegurando que cumplan con el formato requerido.

    Estas herramientas también son útiles para actualizar referencias de forma automática y evitar errores en la citación.
\end{enumerate}

Consejos para la Sección de Referencias

\begin{enumerate}
    \item Mantén un registro desde el inicio: Llevar un registro de cada fuente utilizada durante la investigación facilita mucho el proceso de citación al final.
    \item Verifica la precisión: Revisa que todos los detalles de cada referencia estén correctos (nombre de autores, título, año, etc.).
    \item Evita fuentes poco confiables: Usa únicamente fuentes de calidad, preferiblemente revisadas por pares o de editoriales académicas reconocidas.
    \item Actualiza la lista: A medida que el proyecto avanza, revisa y actualiza las referencias para incluir las fuentes más recientes o relevantes que se hayan agregado.
\end{enumerate}

La sección de referencias valida y respalda los contenidos de una tesis de ingeniería, ofreciendo un marco de credibilidad y facilitando la consulta a futuras investigaciones.
\end{frame}
\begin{frame}[allowframebreaks]{Referencias}
% Despliega las referencias del archivo referencias.bib
  \nocite{*}
  \bibliography{referencias}
  \bibliographystyle{abbrv}
\end{frame}
\end{document}